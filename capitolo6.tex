Nei capitoli precedenti abbiamo esaminato i fondamenti teorici della crittografia
post-quantistica e gli algoritmi standardizzati dal NIST\@. Il periodo 2023--2025
ha segnato però un passaggio decisivo: da casi di studio a implementazioni su scala globale.

Aziende come Google, Apple, Cloudflare e Signal hanno già distribuito protocolli
ibridi ai propri utenti, mentre la NSA ha pubblicato una tabelle di marcia vincolante 
per tutti i sistemi classificati statunitensi. 
Questo capitolo analizza le principali implementazioni reali, la strategia ibrida 
che li accomuna e le sfide emerse in fase di distribuzione.

%----------------------------------------------------------------------------------------
\section[L'approccio ibrido]{L'approccio ibrido: PQC più crittografia classica}

Tutte le implementazioni che discuteremo condividono una scelta architetturale fondatamentale:
non sostituire completamente la crittografia classica on quella pqc, ma parallelizzarle in uno
stato ibrido. La motivazione è bivalente. 

In primo lugo la prudenza scientifica. Gli algoritmi PQC standardizzati nel 2024 sono stati
sottoposti a un processo di revisione pubblica senza precedenti, ma non possono vantare
la stessa profondità di analisi crittoanalitica accumulata a ECC o RSA in decenni di uso. 
L'approccio ibrido quindi garantisce che la sicurezza complessiva non sia inferiore di quella
classica, anche nell'ipotetica eventualità che una vulnerabilità venisse scoperta nei nuovi algoritmi. 

Dall'altro lato, la protezione contro il modello di minaccia \textit{harvest now, decrypt later}
di cui abbiamo parlato nel Capitolo~\ref{chap:Crittografia post quantistica} richiede che i dati siano 
protetti con algoritmi resistenti ai quantistici fin da subito, anche se la minaccia è solo potenziale.
Aggiungendo un layer ML-KEM difianco ad un classico, anche il traffico passato è al sicuro da attacchi, 
poiché un futuro attacante quantistico dovrebbe violare entrambi i protocolli. 

Il costo di questa scelta è un aumento della dimensione dei messaggi e un lieve overhead computazionale, 
che le implementazioni concrete hanno dimostrate essere accettabile nella pratica. 

%----------------------------------------------------------------------------------------
\section{Signal: PQXDH e il Triple Retchet}

Signal è un’app di messaggistica istantanea che ha fatto della privacy e della sicurezza il suo intero scopo di vita
ed infatti è stata la prima a deployare la crittografia post-quantistica su larga scala. Nel settembre 2023, 
la Signal Foundation ha annunciato l'aggiornamento del prodocollo di handshake da X3DH
(\textit{Extended Triple Diffie-Hellman}) a PQXDH
(\textit{Post-Quantum Extended Diffie-Hellman})~\cite{signal_pqxdh}.

\paragraph{Architettura di PQXDH.}
Il protocollo X3DH originale derivava un segreto condiviso combinando quattro scambi Diffie-Hellman su
curve ellittiche. PQXDH aggiunge un ulteriore segreto ottenuto tramite CRYSTALS-Kyber (ora ML-KEM), 
mescolando la chiave inizale tramite una funzione derivazione: 
\[
    SK = \mathrm{KDF}(\,DH_1 \,\|\, DH_2 \,\|\, DH_3 \,\|\, DH_4 \,\|\, SS_{\mathrm{KEM}}\,)
\]
dove $SS_{\mathrm{KEM}}$ è il segreto derivato dall'incapsulamento ML-KEM\@.
La scelta di combinare i due segreti -- piuttosto di sostituire l'uno con l'altro --
implica che un attaccante deve violare sia X25519 che ML-KEM per compromettere la sessione~\cite{signal_pqxdh}.

PQXDH usa Kyber-1024 (livello di sicurezza equivalente ad AES-256) e fornisce
\textit{post-quantum forward secrecy} grazie all'uso di chiavi pre-pubblicate
monouso (\textit{one-time prekeys}) che vengono cancellate dopo l'uso. Tuttavia,
nella prima revisione del protocollo l'autenticazione delle chiavi di identità
restava affidata a ECDSA -- un punto su cui si prevedono evoluzioni future per
resistere anche ad avversari quantistici attivi~\cite{signal_pqxdh}.

\paragraph{Dal Double Ratchet al Triple Ratchet.}
PQXDH protegge la fase di handshake iniziale, ma il \textit{Double Ratchet} --
il meccanismo che fornisce forward secrecy e post-compromise security durante
l'intera conversazione -- rimaneva classico. Per colmare questo gap, Signal ha
successivamente sviluppato il \textit{Triple Ratchet}, che integra un ratchet
post-quantistico basato su ML-KEM a fianco dei due ratchet classici esistenti.
La sfida tecnica principale era mantenere i messaggi di protocollo a dimensioni
comparabili a quelle attuali (circa 36 byte), un vincolo imposto dall'efficienza
della rete~\cite{signal_spqr}.

%----------------------------------------------------------------------------------------
\section{Apple: il protocollo PQ3 per iMessage}

Nel febbraio 2024, Apple ha annunciato PQ3, definendolo come il più significativo
aggiornamento crittografico nella storia di iMessage~\cite{apple_pq3}. PQ3 è
notevole non solo per la sua solidità tecnica, ma anche per la scala di deployment:
centinaia di milioni di dispositivi in tutto il mondo hanno ricevuto automaticamente
il nuovo protocollo con l'aggiornamento a iOS~17.4, iPadOS~17.4, macOS~14.4 e
watchOS~10.4.

\paragraph{La tassonomia dei livelli di sicurezza.}
Apple ha introdotto una propria classificazione per confrontare i protocolli di
messaggistica:
\begin{itemize}
    \item \textbf{Livello 1}: crittografia end-to-end classica, senza protezione
          post-quantistica (es. iMessage precedente al 2024).
    \item \textbf{Livello 2}: PQC applicata solo all'handshake iniziale, senza
          rekeying post-quantistico continuo (es. Signal con PQXDH).
    \item \textbf{Livello 3}: PQC applicata sia all'handshake che a tutta la
          conversazione, con rekeying automatico post-quantistico (PQ3).
\end{itemize}
PQ3 è il primo protocollo di messaggistica ampiamente distribuito a raggiungere
il Livello~3~\cite{apple_pq3}.

\paragraph{Architettura di PQ3.}
Ogni dispositivo genera una coppia di chiavi ML-KEM aggiuntiva durante la
registrazione iMessage, pubblicandola sui server Apple insieme alle chiavi
ECC esistenti. Questo consente al mittente di stabilire un'iniziale chiave
post-quantistica anche quando il destinatario è offline.

Per l'ongoing key establishment, PQ3 implementa un meccanismo di rekeying
periodico integrato nel flusso dei messaggi: nuove chiavi ML-KEM vengono
inviate all'interno della conversazione e usate per derivare chiavi di
cifratura \textit{forward secret}, garantendo che la compromissione di una
chiave in un dato momento non comprometta né i messaggi passati né quelli
futuri. Il rekeying avviene al più ogni 50 messaggi o almeno una volta ogni
sette giorni~\cite{apple_pq3}.

Il protocollo mantiene un approccio ibrido: ECC e ML-KEM coesistono, così che
la sicurezza complessiva non sia mai inferiore a quella di ECC soltanto.
L'autenticazione dell'identità degli utenti rimane affidata a ECDSA, in quanto
i chip \textit{Secure Enclave} dei dispositivi Apple non supportano ancora
algoritmi post-quantistici per la firma -- un'area di sviluppo futuro identificata
da Apple come ``Livello 4''~\cite{apple_pq3_analysis}.

%----------------------------------------------------------------------------------------
\section{Google Chrome e Cloudflare: PQC nel protocollo TLS}

La protezione delle connessioni HTTPS è il caso d'uso con il maggiore impatto
immediato, essendo TLS il protocollo su cui si fonda la sicurezza dell'intero
web. Google e Cloudflare hanno guidato la transizione in questo ambito.

\paragraph{Chrome: da X25519Kyber768 a ML-KEM768.}
Nell'agosto 2023, Google ha abilitato in Chrome~116 il supporto
per l'algoritmo ibrido \texttt{X25519Kyber768}, combinando lo scambio di chiave
classico X25519 con Kyber-768 per proteggere il traffico TLS~1.3 contro
attacchi \textit{store-now-decrypt-later}~\cite{google_chrome_pqc}.

Con Chrome~124 (aprile 2024), il meccanismo è stato abilitato per impostazione
predefinita per tutti gli utenti desktop. Il passaggio ha immediatamente rivelato
un problema di compatibilità pratica: il messaggio TLS \textit{ClientHello},
che normalmente rientrava in un singolo pacchetto TCP, aumentava di dimensione
a causa delle chiavi post-quantistiche più grandi, causando errori di connessione
su server, proxy e firewall che non gestivano correttamente messaggi segmentati su
più pacchetti~\cite{chrome_compatibility}.

Con Chrome~131 (novembre 2024), Google ha completato la transizione allo standard
definitivo, adottando \texttt{ML-KEM768+X25519} (codepoint TLS \texttt{0x11EC}),
in sostituzione del precedente \texttt{Kyber768+X25519} (codepoint \texttt{0x6399}).
Le due varianti sono incompatibili tra loro, rendendo necessario l'aggiornamento
coordinato di client e server~\cite{google_mlkem}.

\paragraph{Cloudflare: deployment su scala globale.}
Cloudflare ha abilitato il supporto per la key agreement post-quantistica sulle
connessioni verso i propri server di origine nel settembre 2023. Entro l'inizio
del 2024, circa il 2\% di tutte le connessioni TLS~1.3 stabilite con Cloudflare
era già protetto con crittografia post-quantistica, grazie principalmente
all'esperimento di Chrome al 10\% degli utenti. Cloudflare prevedeva un'adozione
a due cifre percentuali entro la fine del 2024 con la progressiva estensione
del supporto da parte di Chrome, Firefox e altri browser~\cite{cloudflare_pq2024}.

%----------------------------------------------------------------------------------------
\section{La direttiva NSA: CNSA 2.0}

Sul fronte normativo, la cornice di riferimento più stringente è quella definita
dalla NSA con il \textit{Commercial National Security Algorithm Suite 2.0}
(CNSA~2.0), pubblicato nel settembre 2022~\cite{nsa_cnsa2}.

CNSA~2.0 è vincolante per tutti i sistemi di sicurezza nazionale (NSS) statunitensi
e fissa una tabella di marcia precisa per la migrazione verso la crittografia
post-quantistica, con le seguenti scadenze principali~\cite{nsa_cnsa2}:

\begin{itemize}
    \item \textbf{Immediata}: avvio della transizione per firma di software e
          firmware, con adozione di LMS e XMSS (schemi a firma basati su hash,
          già standardizzati in NIST SP~800-208).
    \item \textbf{Entro il 2025}: supporto agli algoritmi CNSA~2.0 per browser,
          server web e servizi cloud.
    \item \textbf{Dal 2027}: tutti i nuovi sistemi NSS devono essere conformi
          a CNSA~2.0 per impostazione predefinita.
    \item \textbf{Entro il 2031}: algoritmi post-quantistici obbligatori per
          tutte le categorie di sistemi NSS.
    \item \textbf{Entro il 2035}: migrazione completa; gli algoritmi classici
          RSA ed ECC saranno dismessi per tutti i sistemi di sicurezza nazionale.
\end{itemize}

Gli algoritmi approvati da CNSA~2.0 per la fase a regime sono ML-KEM (FIPS~203)
per lo scambio di chiave, ML-DSA (FIPS~204) per le firme digitali, e SLH-DSA
(FIPS~205) come alternativa basata esclusivamente su funzioni hash. In fase di
transizione, la NSA consente soluzioni ibride (classico + PQC) per garantire
interoperabilità, ma chiarisce che l'obiettivo finale è l'eliminazione completa
degli algoritmi vulnerabili~\cite{nsa_cnsa2}.

La direttiva si estende di fatto anche all'industria privata: i contractor della
difesa e i fornitori di prodotti per sistemi classificati sono tenuti ad allinearsi
alle stesse scadenze come requisito contrattuale, creando un effetto di propagazione
verso l'intero settore tecnologico.

%----------------------------------------------------------------------------------------
\section{Sfide pratiche dell'adozione}

L'esperienza delle implementazioni reali ha evidenziato alcune sfide ricorrenti
che vanno oltre la mera sostituzione algoritmica.

\paragraph{Dimensione delle chiavi e dei messaggi.}
Gli algoritmi PQC producono chiavi e ciphertext significativamente più grandi
rispetto ai loro equivalenti classici. Come osservato con Chrome, una chiave
pubblica ML-KEM è dell'ordine di 1~KB, contro i 32~byte di X25519. Questo
può frammentare messaggi di protocollo che tipicamente si assumevano di piccola
dimensione, causando incompatibilità con implementazioni rigide di TLS, firewall
e middleware

\paragraph{Overhead computazionale.}
Le operazioni ML-KEM sono computazionalmente più costose delle operazioni ECDH,
ma l'overhead nella pratica è limitato: studi empirici mostrano che un handshake
TLS ibrido aggiunge tipicamente 1--2 millisecondi di latenza, un costo accettabile
per la maggior parte delle applicazioni. Il principale impatto si avverte invece
su dispositivi con risorse limitate (IoT, smart card, hardware embedded), dove
la disponibilità di implementazioni ottimizzate diventa critica.

\paragraph{Crypto-agility.}
La transizione in corso ha rafforzato il concetto di \textit{crypto-agility}:
la capacità di un sistema di aggiornare i propri algoritmi crittografici senza
richiedere una riprogettazione completa. La migrazione da Kyber a ML-KEM in Chrome
-- algoritmi quasi identici ma con codepoint TLS incompatibili -- è un esempio
concreto di quanto sia importante progettare i sistemi in modo da separare
l'interfaccia crittografica dalla sua implementazione~\cite{google_mlkem}.

\paragraph{Compatibilità e ossificazione dei protocolli.}
Cambiamenti nello stack crittografico si scontrano spesso con la cosiddetta
\textit{ossificazione dei protocolli}: l'insieme di assunzioni implicite -- spesso
scritte nel codice di middleware, proxy e firewall -- su dimensioni, formati e
comportamenti dei pacchetti. Il caso Chrome~124 ha dimostrato che anche un
cambiamento ben pianificato e segnalato con anticipo può causare interruzioni
significative in produzione, sottolineando l'importanza di un rollout graduale e
di strumenti di test pubblici~\cite{chrome_compatibility, cloudflare_pq2024}.

La Tabella~\ref{tab:impl_confronto} riassume le principali implementazioni
analizzate in questo capitolo.

\begin{table}[ht]
\centering
\caption{Confronto tra le principali implementazioni industriali di PQC (2023--2024).}
\label{tab:impl_confronto}
\begin{tabular}{lllll}
\hline
\textbf{Attore} & \textbf{Protocollo} & \textbf{Algoritmo} & \textbf{Contesto} & \textbf{Anno} \\
\hline
Signal      & PQXDH         & ML-KEM-1024 + X25519  & Messaggistica E2EE & 2023 \\
Google      & TLS ibrido    & ML-KEM-768 + X25519   & HTTPS (Chrome)     & 2023 \\
Cloudflare  & TLS ibrido    & ML-KEM-768 + X25519   & CDN / TLS globale  & 2023 \\
Apple       & PQ3           & ML-KEM + ECDH (P-256) & iMessage           & 2024 \\
NSA         & CNSA 2.0      & ML-KEM, ML-DSA,       & Sistemi classificati & 2022 \\
            &               & SLH-DSA, LMS/XMSS    &                    &      \\
\hline
\end{tabular}
\end{table}

Una considerazione finale che possiamo trarre da tutte queste implementazioni è che la scelta
ibrida non è una soluzione temporanea ma una strategia deliberata per garantire sicurezza a lungo termine. 
Nessuna delle implementazioni che abbiaamo analizzato prevede di rimuovere completamente la crittografia
classica, preferendo invece costruire un livello di protezione aggiuntivo che sia conservativo rispetto 
all'incertezza crittoanalitica. 